\documentclass{article}
\usepackage[latin1]{inputenc}
\usepackage[T1]{fontenc}
\usepackage{color}
\definecolor{blue}{cmyk}{1,1,0,0}
\definecolor{red}{cmyk}{0,1,1,0}
\definecolor{green}{cmyk}{1,0,1,0}
\definecolor{green}{rgb}{0,0.8,0}
\definecolor{lightgreen}{rgb}{0.6,1,0.6}
\definecolor{gray}{rgb}{0.33, 0.33, 0.33}
\definecolor{forestgreen}{rgb}{0.1,0.6,0.1}
\definecolor{darkgreen}{rgb}{0,0.6,0}
\definecolor{lightblue}{rgb}{0.5,0.5,1}
\definecolor{purple}{rgb}{1,0,1}
\definecolor{lightred}{rgb}{1,0.5,0.5}
\definecolor{darkred}{rgb}{0.6,0,0}
\definecolor{magenta}{cmyk}{0,1,0,0}
\definecolor{yellow}{cmyk}{0,0,1,0}
\definecolor{cyan}{cmyk}{1,0,0,0}
\definecolor{brown}{rgb}{0.3,0.3,0.8}
\definecolor{orange}{named}{Salmon}
\definecolor{classcolor}{named}{Peach}
\definecolor{libcolor}{named}{MidnightBlue}
\definecolor{commentcolor}{named}{Melon}
\definecolor{kwdcolor}{named}{Tan}
\definecolor{stringcolor}{rgb}{0.1,0.6,0.1}
\usepackage{listings}
\lstavoidwhitepre
\lstset{language=caml}
\def\bfcolor#1{\ttfamily\bfseries\color{#1}}
\def\itcolor#1{\itshape\color{#1}}
\newcommand{\kwd}{\bfcolor{kwdcolor}}
\newcommand{\ndkwd}{\bfcolor{libcolor}}
\newcommand{\emphstyle}{\bfcolor{classcolor}}
\lstdefinestyle {javastyle}{
   keywordstyle=\kwd,
   ndkeywordstyle=\ndkwd,
   commentstyle=\color{commentcolor}\scshape,
   emphstyle=\emphstyle,
   emphstyle={[2]\color{lightred}},
   stringstyle=\color{stringcolor}\itshape,
   moredelim=[s][keywordstyle]{(*+}{+*)}
}
\lstset{style=javastyle}
\lstset{morekeywords={[2]Random,Sys,Printf,Array}}
\let\lst\lstinline
\begin{document}
\title{Recursive fair? shuffle}
\author{}
\maketitle
\lstset{emph={merge,shuffle}}%
\lstset{emph=[2]{gen,split}}
Significant functions are \lst|merge| and~\lst|shuffle|.
Functions \lst|gen| and \lst|split| are also worth a look.
\begin{lstlisting}[escapechar=@,mathescape=true,numbers=left,stepnumber=5,numberstyle=\color{brown},frame=tb,rulecolor=\color{kwdcolor}]
(*+@\label{start}@
  Special comment!
   Start at line @\ref{start}@.
   End at line @\ref{over}@.
+*)

(* Read arguments, $n$ is list len, � nexp � is number of experiments *)
let n =
  if Array.length Sys.argv > 1 then int_of_string Sys.argv.(1)
  else 10

let nexp =
  if Array.length Sys.argv > 2 then int_of_string Sys.argv.(2)
  else 100

(* Generate list $[m\ldots{}n]$ *)
let rec gen m n =
  if m <= n then m::gen (m+1) n
  else []

(* Merge (uniformly) at random *)
let rec merge r xs sx ys sy = match xs, ys with
| [],[] -> r
| x::rx, [] -> merge (x::r) rx (sx-1)  ys sy
| [],y::ry -> merge (y::r) xs sx ry (sy-1)
| x::rx, y::ry ->
    if Random.int(sx+sy) < sx then
      merge (x::r) rx (sx-1) ys sy
    else
      merge (y::r) xs sx ry (sy-1)

(* Split a list into two lists of same size *)
let rec do_split even se odd so = function
  | [] -> (even,se), (odd,so)
  | [x] -> (x::even,se+1), (odd,so)
  | x::y::rem -> do_split (x::even) (se+1) (y::odd) (so+1) rem

let split xs = do_split [] 0 [] 0 xs

(* Actual suffle *)
let rec shuffle xs = match xs with
| []|[_] -> xs
| _ ->
    let (ys,sy), (zs,sz) = split xs in
    merge [] (shuffle ys) sy (shuffle zs) sz



(* Perform experiment *)
let zyva n m =
  let xs = gen 1 n in
  let t = Array.make_matrix (n+1) (n+1) 0 in
  for k = 1 to m do
    let ys = shuffle xs in
    let idx = ref 1 in
    List.iter
      (fun x -> t.(!idx).(x) <-  t.(!idx).(x) + 1 ; incr idx)
      ys
  done ;
  let mm = float_of_int m in
  for i = 1 to n do
    let t = t.(i) in
    for k=1 to n do
      let f = float_of_int t.(k) *. 100.0 /. mm in
      printf " %02.2f" f 
    done ;
    print_endline ""
  done

let _ = zyva n nexp ; exit 0

(* @\label{over}@Start at line @\ref{start}@, end here. *)

(*
(***************)
(* Print lists *)
(***************)
*)

open Printf

let rec do_plist = function
  | [] -> printf "}"
  | x::xs -> printf ", %i" x ; do_plist xs

let plist = function
  | [] -> printf "{}"
  | x::xs -> printf "{%i" x ; do_plist xs

let plistln xs = plist xs ; print_endline ""
\end{lstlisting}
\end{document}
